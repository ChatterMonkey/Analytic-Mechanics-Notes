\documentclass{homework}
\author{Maya Basu}
\class{Analytic Mechanics}
\title{Class Notes}

\newcommand{\inte}{\text{Int}}
\newcommand{\clos}[1]{\overline{#1}}
\newcommand{\RR}{\mathbb{R}}
\newcommand{\BB}{\mathbb{B}}
\newcommand{\OO}{\mathbb{O}}


\begin{document} \maketitle

\section{Kinematics}
\begin{enumerate}
    \item A body remains in uniform motion unless a force acts on it
    \item $\frac{d\vec{p}}{dt} = \overarrow{F}$ 
    \item For central forces, two bodies exert equal and opposite forces on one another
\end{enumerate}
Non-relatavistic momntum: $\vec{p} = m_{inertial}\vec{v}$ where $m_{inertial}$ is the inertial mass.

\[\frac{d\vec{p}}{dt} = m_i\frac{d\vec{v}}{dt} = m_i\vec{a}\]
so
\[m_i\vec{a} = \vec{F} = m_gg\]
where $m_g$ is the gravitational mass. It is an experimental fact that $m_i = m_g$, there is no difference between gravity and acceleration, the "Principle of equivelence"


If $\vec{r}(t)$ is the position of a point then $\frac{\partial r}{\partial t} = \frac{dr}{dt} + O(dt)$. Specifically, $\partial r = \vec{r}(t + \partial t) - \vec{r}(t)$ and expanding with the taylor series of $\vec{r}$ about $t$ gives $\partial r = \vec{r}(t) + \vec{r'}\partial t - \vec{r}(t)$ or $\frac{\partial \vec{r}}{\partial t} = \vec{r'}$.

Similarly, $\vec{a} = \frac{d\vec{v}}{dt}$.

In a cartesian system, $\vec{r} = x(t)\hat{i} + y(t)\hat(j) + z(t)\hat{k}$, $\vec{v} = \dot{x}(t)\hat{i} + \dot{y}(t)\hat(j) + \dot{z}(t)\hat{k}$, and  $\vec{a} = \ddot{x}(t)\hat{i} + \ddot{y}(t)\hat(j) + \ddot{z}(t)\hat{k}$.

However, the axis do not have to be fixed. 

Suppose we have vectors $\vec{e_r}$ pointing to the position of the particle and $\vec{e_{\theta}}$ pointing at a right angle to the former. Then in terms of cartesian coordinates we have
\[\hat{e_r} = \cos(\theta)\hat i + \sin (\theta) \hat j\]
\[\hat{e_{\theta}} = -\sin(\theta)\hat i + \cos (\theta) \hat j\]
And we have $\vec{v}(t) = \frac{d}{dt}(r\hat e_r) = \dot r \hat e_r + r \frac{d}{dt}\hat e_r $. However, since $\vec{e_r} = \cos(\theta)\hat i + \sin (\theta) \hat j$, we know that $\frac{d}{dt}\vec{e_r} = (-\sin(\theta)\hat i + \cos (\theta) \hat j)\dot{\theta} = \hat{e_{\theta}}\dot{\theta}$. Substituting back we get 
\[\vec{v}(t) = \dot r \hat e_r + \hat{e_{\theta}}r\dot{\theta}\]

Differentiating again gives 
\[\vec{a}(t) = (\ddot{r} - r\dot{\theta}^2)\hat e_r + (2\dot{r}\dot{\theta} + r\ddot{\theta})\hat{e_{\theta}})\]
which can alternatly be written as
\[\vec{a}(t) = (\ddot{r} - r\dot{\theta}^2)\hat e_r + \frac{1}{r}\frac{d}{dt}(r^2\dot{\theta})\hat{e_{\theta}})\]


\section{Cylindrical Coordinates}

\section{Motion}
\subsection{Partical undergoing vertical motion, only under the force of gravity (no air resistance)}
\[m_i\frac{d\vec {v}}{dt} = -m_g g\]
$m_i = m_g$ experimentally.
\[v = C - gt\]
If $v = v_0$ are some time $t_0$ then
\[v = v_0 - gt\]
Integrating again gives
\[z = z_0 + v_0t - \frac{1}{2}gt^2\]
\subsection{Air resistance (linear in the speed)}
\[m\frac{d\vec {v}}{dt} = -m\alpha v-m g\]
(we include the $m$ factor in front of $\alpha$ for convinience making $\alpha$ the perunit mass value.
\[\frac{d\vec {v}}{dt} +\alpha v = - g\]
(Inhomogeneous linear differential equation). The general solution is $v(t) = -\frac{g}{\alpha} + Ce^{-\alpha t}$ which we can find by multiplying byt eh appropriate exponential, or alternatly by sepparating variables. If $v = 0$ at $t = 0$, $v(t) = -\frac{g}{\alpha}(1 - e^{-\alpha t})$. We see at $t$ increases, this expression approaches $-\frac{g}{\alpha}$ which is the terminal velocity.

We get the displacment, $z(t)$  by integrating again to get the general solution $z(t) = -\frac{g}{\alpha} t - \frac{g}{\alpha^2}e^{-\alpha t} + C$. If $z(0) = 0$ we get
\[z(t) = -\frac{g}{\alpha}(t - \frac{1}{\alpha}(1 -  e^{-\alpha t})\]
We note that as $t$ increases this becomes linear because of the terminal velocity. (Check this expression).

\section{Quadratic Air Resistance}

For a particle falling down:
\[m\frac{dv}{dt} = m\alpha v^2 - mg\]
We would need a $-$ sign infront of the air resistance factor if the particle was being thrown upwards.

We can sepparate this equation and solve by  decomposing the fraction into two fractions with simple denominators, (check this) giving
\[v(t) = v_{\infty} \frac{1 - e^{-2\sqrt{\alpha g }t}}{1 + e^{-2\sqrt{\alpha g }t}}\]
As $t \rightarrow \infty$, the exponential factors decay and $v(t) \rightarrow v_{\infty}$.

Question: What is the trick for sepparating out a fraction without cross multiplying?


\section{2D motion with no air resistance}

because there is no air resistance, we can decouple the two dimentions of motion:
\[z(t) = v\sin(\alpha)-\frac{1}{2}gt^2\]
For a particle startign at the origin, shot with speed $v$ at angle $\alpha$, and 
\[x(t) = v\cos (\alpha)t\]
The particle is at the ground at $z = 0$, the solutions for which are $t = 0$ and $t = \frac{2v\sin (\alpha )}{g}$. At this point the particle has traveled a distance $v\cos (\alpha ) \frac{2v\sin (\alpha )}{g} = \frac{v^2}{g}\sin (2 \alpha )$ which is maximized when $\alpha = 45^{\circ}$.

\section{Motion along a frictionless inclined plane}

Suppose that $\hat\sigma$ points out of the plane and $\hat s$ point up along the plane. Then along the plane
\[m\ddot s = -mg\sin (\alpha)\]
Perpendicular to the plane:
\[m \ddot {\sigma} = N - mg \cos (\alpha) =0\]
Thus if we assume the object is realsed at rest from $s = 0$ then
\[\dot s = -g \sin (\alpha) t\]
\[s = \frac{1}{2}\g \sin g (\alpha)t^2\]

\section{Motion along an inclined plane with friction}

The friction depends on the weight of the box, $F = \mu N$, in this case,
\[\ddot s = -g(1 - \mu \cot (\alpha)\]
or the same motion as before but with gravity decreased to $g(1 - \mu \cot (\alpha))$.



\section{Free Space Rocket Motion}
Motion in free space (no forces including gravity)
In time $dt$, the rocket expells $dm'$ with a velocity $u$ relative to the rocket. By the second law, $\dot p = 0$ so 
\[p(t) = p(t + dt) \rightarrow mv = (m - dm')(v + dv) + dm'(v - u)\]
Multiplying this out, setting $dm = -dm'$ and neglecting the $dm'dv$ term, we get
\[0 = mdv + udm\]
or
\[m\frac{dv}{dt} = -u\frac{dm}{dt}\]
so for negative decrease in mass ($\frac{dm}{dt} <0$) we get positive acceleration. If $u$ is constant then 
\[v - v_0 = u\ln \left( \frac{m_0}{m} \right)\]
where we are integrating from $m_0$ to $m$. 

\section{Rocket Motion with Gravity}
$p$ is no longer conserved
\[p(t + dt) - p(t) = -mgdt\]
\[mdv + udm = -mgdt\]
\[\int dv = -g\int dt - \int u \frac{dm}{m}\]
If $v = 0$, $u$ is constant, and $m = m_0$ t $t = 0$, this becomes 
\[v = -gt -u\ln \left(\frac{m}{m_0} \right) \]
If the mass of the rocket is lost at a consta rate $\alpha = - \frac{dm}{dt}$ then $m = m_0(1 - \alpha t)$ so we get
\[v = -gt -u\ln(1 - \alpha t) \]
the taylor expansion of which gives
\[\]





\section{Conservation of Energy}




Suppose that 



We 





$\vec{r}(t) = r(t)\hat{e_r}(t)$, with $\hat e_{\theta}$ pointing in the direction of motion.
\[e_r = cos(\theta)\hat i\]


\end{document}