\documentclass{homework}
\author{Maya Basu}
\class{Analytic Mechanics}
\title{Class Notes}

\newcommand{\inte}{\text{Int}}
\newcommand{\clos}[1]{\overline{#1}}
\newcommand{\RR}{\mathbb{R}}
\newcommand{\BB}{\mathbb{B}}
\newcommand{\MM}{\mathbb{M}}
\newcommand{\OO}{\mathbb{O}}
\newcommand{\m}[1]{\begin{bmatrix} #1 \end{bmatrix}}


\begin{document} \maketitle
Constants
k = coulumb force constant
\section{Kinematics}
\begin{enumerate}
    \item A body remains in uniform motion unless a force acts on it
    \item $\frac{d\vec{p}}{dt} = \overarrow{F}$ 
    \item For central forces, two bodies exert equal and opposite forces on one another
\end{enumerate}
Non-relatavistic momntum: $\vec{p} = m_{inertial}\vec{v}$ where $m_{inertial}$ is the inertial mass.

\[\frac{d\vec{p}}{dt} = m_i\frac{d\vec{v}}{dt} = m_i\vec{a}\]
so
\[m_i\vec{a} = \vec{F} = m_gg\]
where $m_g$ is the gravitational mass. It is an experimental fact that $m_i = m_g$, there is no difference between gravity and acceleration, the "Principle of equivelence"


If $\vec{r}(t)$ is the position of a point then $\frac{\partial r}{\partial t} = \frac{dr}{dt} + O(dt)$. Specifically, $\partial r = \vec{r}(t + \partial t) - \vec{r}(t)$ and expanding with the taylor series of $\vec{r}$ about $t$ gives $\partial r = \vec{r}(t) + \vec{r'}\partial t - \vec{r}(t)$ or $\frac{\partial \vec{r}}{\partial t} = \vec{r'}$.

Similarly, $\vec{a} = \frac{d\vec{v}}{dt}$.

In a cartesian system, $\vec{r} = x(t)\hat{i} + y(t)\hat(j) + z(t)\hat{k}$, $\vec{v} = \dot{x}(t)\hat{i} + \dot{y}(t)\hat(j) + \dot{z}(t)\hat{k}$, and  $\vec{a} = \ddot{x}(t)\hat{i} + \ddot{y}(t)\hat(j) + \ddot{z}(t)\hat{k}$.

However, the axis do not have to be fixed. 

Suppose we have vectors $\vec{e_r}$ pointing to the position of the particle and $\vec{e_{\theta}}$ pointing at a right angle to the former. Then in terms of cartesian coordinates we have
\[\hat{e_r} = \cos(\theta)\hat i + \sin (\theta) \hat j\]
\[\hat{e_{\theta}} = -\sin(\theta)\hat i + \cos (\theta) \hat j\]
And we have $\vec{v}(t) = \frac{d}{dt}(r\hat e_r) = \dot r \hat e_r + r \frac{d}{dt}\hat e_r $. However, since $\vec{e_r} = \cos(\theta)\hat i + \sin (\theta) \hat j$, we know that $\frac{d}{dt}\vec{e_r} = (-\sin(\theta)\hat i + \cos (\theta) \hat j)\dot{\theta} = \hat{e_{\theta}}\dot{\theta}$. Substituting back we get 
\[\vec{v}(t) = \dot r \hat e_r + \hat{e_{\theta}}r\dot{\theta}\]

Differentiating again gives 
\[\vec{a}(t) = (\ddot{r} - r\dot{\theta}^2)\hat e_r + (2\dot{r}\dot{\theta} + r\ddot{\theta})\hat{e_{\theta}})\]
which can alternatly be written as
\[\vec{a}(t) = (\ddot{r} - r\dot{\theta}^2)\hat e_r + \frac{1}{r}\frac{d}{dt}(r^2\dot{\theta})\hat{e_{\theta}})\]


\section{Cylindrical Coordinates}

\section{Motion}

\subsection{Friction less motion}

\subsubsection{Vertical motion (no air resistance)}
\[m_i\frac{d\vec {v}}{dt} = -m_g g\]
$m_i = m_g$ experimentally.
\[v = C - gt\]
If $v = v_0$ are some time $t_0$ then
\[v = v_0 - gt\]
Integrating again gives
\[z = z_0 + v_0t - \frac{1}{2}gt^2\]



\subsubsection{2D motion (No air resistance)}

because there is no air resistance, we can decouple the two dimentions of motion:
\[z(t) = v\sin(\alpha)-\frac{1}{2}gt^2\]
For a particle startign at the origin, shot with speed $v$ at angle $\alpha$, and 
\[x(t) = v\cos (\alpha)t\]
The particle is at the ground at $z = 0$, the solutions for which are $t = 0$ and $t = \frac{2v\sin (\alpha )}{g}$. At this point the particle has traveled a distance $v\cos (\alpha ) \frac{2v\sin (\alpha )}{g} = \frac{v^2}{g}\sin (2 \alpha )$ which is maximized when $\alpha = 45^{\circ}$.


\subsubsection{Inclined plane (Friction Free)}

Suppose that $\hat\sigma$ points out of the plane and $\hat s$ point up along the plane. Then along the plane
\[m\ddot s = -mg\sin (\alpha)\]
Perpendicular to the plane:
\[m \ddot {\sigma} = N - mg \cos (\alpha) =0\]
Thus if we assume the object is realsed at rest from $s = 0$ then
\[\dot s = -g \sin (\alpha) t\]
\[s = \frac{1}{2}\g \sin g (\alpha)t^2\]


\section{Frictionfull motion}



\section{Motion along an inclined plane with friction}

The friction depends on the weight of the box, $F = \mu N$, in this case,
\[\ddot s = -g(1 - \mu \cot (\alpha)\]
or the same motion as before but with gravity decreased to $g(1 - \mu \cot (\alpha))$.



\subsection{Linear and quadratic air resistance}
If we approximate air resistance as $\approx a + bv + cv^2 + \cdots$ and take the first few terms to be a resonable apporximation for small $v$, then since we want $f_{friction} = 0$ at $v = 0$, we get $bv + cv^2$. For large objects such as a baseball we can generally neglect $f_{lin}$ whereas for objects like a water drop both terms are importaint. However in experiments such as the oil drop experiments $f_{lin}$ dominates

\section{Linear air resistance}


\section{2D motion with linear air resistance}
The general setup is
\[m\vec{\dot{v}} = -m\vec{g} -b\vec{v}.\]
Notice that this only depends on $v$, not $r$ so the equation is first order (then we can integrate to get $r$. Additionally, we can sepparate the equation into components which is only possible because the drag is not quadratic. 

\subsection{Horizontal motion with linear friction}

If we sepparate out the horizontal component we get
\[\dot{v_x} = -\frac{b}{m}v_x\]
or $v_x = v_{0x}e^{-\frac{b}{m}t}$ and $x(t) = x(0) + \int_{0}^{t}v_{0x}e^{-\frac{b}{m}t'}dt' = v_{0x}\frac{m}{b}(1 - e^{-t\frac{b}{m}})$ where $v_{0x}\frac{m}{b}$ is the limit.

\subsection{Vertical motion with linear friction}
Consider a particle with just vertical motion
\[m\frac{dv}{dt} = -m\alpha v-m g\]
(we include the $m$ factor in front of $\alpha$ for convinience making $\alpha$ the perunit mass value.
\[\frac{d\vec {v}}{dt} +\alpha v = - g\]
(Inhomogeneous linear differential equation). The general solution is $v(t) = -\frac{g}{\alpha} + Ce^{-\alpha t}$ which we can find by multiplying byt eh appropriate exponential, or alternatly by sepparating variables. If $v = 0$ at $t = 0$, $v(t) = -\frac{g}{\alpha}(1 - e^{-\alpha t})$. We see at $t$ increases, this expression approaches $-\frac{g}{\alpha}$ which is the terminal velocity.

We get the displacment, $z(t)$  by integrating again to get the general solution $z(t) = -\frac{g}{\alpha} t - \frac{g}{\alpha^2}e^{-\alpha t} + C$. 
\[z(t) = (v_{yo}+v_{ter})\tau(1 - e^{-t/\tau}) - v_{ter}t\]
where $\tau = \alpha$ (which would usually be $\tau = b/m$)
If $z(0) = 0$ we get
\[z(t) = -\frac{g}{\alpha}(t - \frac{1}{\alpha}(1 -  e^{-\alpha t})\]
We note that as $t$ increases this becomes linear because of the terminal velocity. (Check this expression).

\section{Quadratic Air Resistance}

Unfortunately unlike the linear case if the motion is in 2d we can not separate the equations and need to solve numerically. However, when the motion is restricted to 1d we can solve this problem. 

\subsection{Horizontal motion with quadratic drag}

If an object is moving horizontally with only quadratic drag we can write
\[m\frac{dv}{dt} = -cv^2\]
We sepparate this equation to get
\[m\int{0}^{t} v^{-2}dv= -c \int{0}^{t}dt]\]
\[mv^{-1}|{0}^{t} = ct]\]
\[m(\frac{1}{v_f} - \frac{1}{v_i}) = ct\]
or
\[\frac{1}{v_f}  = ct/m + 1/v_i\]
\[v_f = \frac{v_i}{ctv_i/m + 1}\]

To find the displacement we can integrate again to get
\[\Delta x = v_i\frac{m}{cv_i}\ln{ctv_i/m + 1}\]


\subsection{Vertical motion with quadratic drag}

For a particle falling down:
\[m\frac{dv}{dt} = m\alpha v^2 - mg\]
We would need a $-$ sign infront of the air resistance factor if the particle was being thrown upwards. When this equation is ballanced we can see that the terminal velocity $v_{\infty} = \sqrt{mg}{c}$. 

We can sepparate this equation and solve by  decomposing the fraction into two fractions with simple denominators, (check this) giving
\[v(t) = v_{\infty} \frac{1 - e^{-2\sqrt{\alpha g }t}}{1 + e^{-2\sqrt{\alpha g }t}}\]
As $t \rightarrow \infty$, the exponential factors decay and $v(t) \rightarrow v_{\infty}$.

Question: What is the trick for sepparating out a fraction without cross multiplying?




\section{Free Space Rocket Motion}
Motion in free space (no forces including gravity)
In time $dt$, the rocket expells $dm'$ with a velocity $u$ relative to the rocket. By the second law, $\dot p = 0$ so 
\[p(t) = p(t + dt) \rightarrow mv = (m - dm')(v + dv) + dm'(v - u)\]
Multiplying this out, setting $dm = -dm'$ and neglecting the $dm'dv$ term, we get
\[0 = mdv + udm\]
or
\[m\frac{dv}{dt} = -u\frac{dm}{dt}\]
so for negative decrease in mass ($\frac{dm}{dt} <0$) we get positive acceleration. If $u$ is constant then 
\[v - v_0 = u\ln \left( \frac{m_0}{m} \right)\]
where we are integrating from $m_0$ to $m$. 

\section{Rocket Motion with Gravity}
$p$ is no longer conserved
\[p(t + dt) - p(t) = -mgdt\]
\[mdv + udm = -mgdt\]
\[\int dv = -g\int dt - \int u \frac{dm}{m}\]
If $v = 0$, $u$ is constant, and $m = m_0$ t $t = 0$, this becomes 
\[v = -gt -u\ln \left(\frac{m}{m_0} \right) \]
If the mass of the rocket is lost at a consta rate $\alpha = - \frac{dm}{dt}$ then $m = m_0(1 - \alpha t)$ so we get
\[v = -gt -u\ln(1 - \alpha t) \]
the taylor expansion of which gives
\[v(t) = -gt + u\alpha t + O(t^2)\]
so $v > 0$ provided that $\frac{u\alpha}{g} > 1$ (check this).

\section{Conservation Laws}

\begin{enumerate}
    \item {Conservation of linear momentum $\vec{\dot{p}} = \vec{F}$ so $p = mv$ is constant componentwise if the forces are $0$ along that direction}
    \item {Conservation of angular momentum relative to a point, $\vec{L} = \vec{r} \times \vec{p}$, and $\vec{\dot{L}} = \vec{\dot {r}} \times \vec{p} \text{(this factor is $0$ (parallel vectors))} + \vec{r} \times \vec{\dot {p}} = \vec{r} \times \vec{F} = \Tau$. So if $\Tau = 0$ then the torque about that point is constant}
    \item {Energy Conservation - Work done by $F$ in moving a particle of mass $M$ from $r$ to $r + dr$ is $\vec{F} \cdot \vec{dr} = m\frac{dv}{dt}\frac{dr}{dt} dt = m\frac{dv}{dt}vdt = m$}
    
\end{enumerate}



\section{Potential Energy}
A scalar potential energy function can not be defined for all forces, for example, magnetic, friction, and constrained forces

An irrotational vector field is a field with $0$ curl, $\nabla \times f = 0$. 

Suppose $\vec{f}(\vec{x})$ is irrotational in a simply connected (any closed curve and be continuously defomed to a loint without leaving $V$) volume $V$. Then there exists a scalar function $U(x)$ such taht $\vec{f} = - \nabla U(\vec{x})$.

If the volume is not simply connected, then a curve around a hole in the volume can have a non zero curl traveling around the hole, since we can not use stokes theorem. We disregard this case since we want the path integrals in this area to only depend on the initial and final points, giving 
\[W = T_1 - T_0 = -(U_1 - U_0)\]
is the change in potential. If we define $E = T + U$ then $E_1 = E_0$. 

Remark: $dU = U(\vec{x} + d\vec{x}) - U(\vec{x}) = -\int_{\vec{x}}^{\vec{x} + \vec{dx}}\vec{f}\cdot dx$. Since all path integral between these two points must be the same, we picka s traight line, giving $\nabla U \cdot dx = -\vec{f}\cdot d\vec{x}$. Since this is alwyas true we conclude $\vec{f} = - \nabla U$. 


Suppose that $\vec{f}$ is a time dependant irrotational foce field. Then
\[\frac{d\vec{E}}{dt} = \frac{d}{dt}\frac{1}{2}m\vec{v}^2 + \frac{d}{dt}U(\vec{x},t)  = m\vec{v}\vec{\dot{v}} + \nabla U \vec{\dot{x}} + \frac{\delta U}{\delta t} = \vec{v} (-\nabla U) + \nabla U \vec{v} + \frac{\delta U}{\delta t}\]
So if $\frac{\delta U}{\delta t} \neq 0$ then energy is no longer conserved.

\subsection{Consequences of energy conservation in one dimention}
For 1D we have $\frac{1}{2}mv^2 + U(x) = E$ so $v = \pm \sqrt{\frac{2(E-U(x)}{m}}$. To find the displacment we can sepparate variables to get





\section{Center of Mass}

We define the center of mass (CM) to be at
\[\vec{R} = \frac{1}{M}\sum_{\alpha}m_{\alpha}\vec{r}_{\alpha}\]
or
\[\vec{R} = \frac{1}{M} \int \phi \vec{r} dv\]
The center of mass is a property of the system of particles, if we displace the origin by $\vec{R} = \vec{a} + \vec{R'}$, $\vec{r}_{\alpha} = \vec{a} + \vec{r'}_{\alpha}$, then $\vec{R'} = \frac{1}{M}\sum_{\alpha}m_{\alpha}\vec{r'}_{\alpha} = \frac{1}{M}\sum_{\alpha}m_{\alpha}(\vec{r} - \vec{a})_{\alpha} = \frac{1}{M}\sum_{\alpha}m_{\alpha}\vec{r}_{\alpha} -a\frac{1}{M}\sum_{\alpha} m_{\alpha}$
So this is the same point in the distribution and we can always pick the origin at the ceneter of mass if we want $\sum_{\alpha}m_{\alpha}\vec{r'}_{\alpha} = 0$


For a continuous mass distribution, $M = \int_{V} \rho(\vec{r'})d^3r'$ and $ \int_{V} \vec{r'}\rho(\vec{r'})d^3r' = 0$ when the origin is the center of mass

\subsection{Momentum with the center of mass}

The momentum of a multiparticle system is:

\[P = \sum_{a}p_a =\sum_{a}m_a\vec{\dot{r}} = M\dot{\vec{R}} \]
So the total momentum of a system of particles is the same as the mumentum of a single ficticious particle with mas $M$ and the same velocity as the CM. Taking the derivative we get
\[\dot{} = F_{\text{net ext}} = M\ddot{\vec{R}} \]
So the center of mass moves as if it was a ficticious particle of mass M with just the external forces acting on it.

\section{Linear Momentum}

The force on paticle $\alpha$ is 
\[\vec{f}_{\alpha} = \vec{F}_{\alpha}^{ext} + \sum_{\beta, \beta \neq \alpha}\vec{f}_{\beta}^{int}\]
We assume the weak version of Newton's thrid law, that $\vec{f}_{\alpha\beta}^{int} = \vec{f}_{\beta\alpha}^{int}$ (which does not require the forces to be central). Then the linear momentum is
\[\vec{\dot{p_{\alpha}}} = \vec{F}_{\alpha}^{ext} + \sum_{\beta, \beta \neq \alpha}\vec{f}_{\beta}^{int}\]
Considering the whole system
\[\vec{\dot{P}} = \sum_{\alpha} \vec{\dot{p_{\alpha}}} = \sum_{\alpha}(\vec{F}_{\alpha}^{ext} + \sum_{\beta, \beta \neq \alpha}\vec{f}_{\beta}^{int})\]
\[\vec{\dot{P}} = \sum_{\alpha} \vec{\dot{p_{\alpha}}} = \sum_{\alpha}\vec{F}_{\alpha}^{ext} = \vec{F_{\text{net ext}} }\]

\section{Angular Momentum}

The angular momentum of a single particle with respect to a certain origin is defined as 
\[l = \vec{r} \times \vec{p}\]

Taking the  time derivative gives
\[\dot{l} = \vec{\dot{r}} \times \vec{p} + \vec{r} \times \vec{\dot{p}}\]
and the first term is $0$ so we get
\[\dot{l} = \vec{r} \times F_{net} = T\]

\subsection{Angular momentum with several particles}

The total angular momentum is 
\[L = \sum_{a}r_a \times p_a\]
Differentiating with respect to $t$ we get
\[\dot{L} = \sum_{a}\dot{r}_a \times p_a + \dot{r}_a \times \dot{p}_a = \sum_{a}\dot{r}_a \times F_a\]
or pairing internal terms, if the force is central and satisfies the third law, we get
\[\dot{L} = T_{ext}\]


\section{Moment of Inertia}



\section{Kepler's Laws}

\subsection{Kepler's second law}

"a planet orbiting a sun sweeps out equal areas in equal times" or $\dot{A}$ is constant. The area between the position vectors is $\frac{1}{2}(r_1 \times r_2)$ or
\[dA = \frac{1}{2}(r \times v dt)\]
\[dA/dt = \frac{1}{2m}(r \times p) = \frac{1}{2m}l\]
and since agular momentum is conserved, $\dot{A} $ is constant


\section{Oscilations}

Suppose that a particle is at equilibrium at $x_0$. If it is perturbed away from $x_0$ slightly we get oscilitory motion. Let $x$ be the distance from $x_0$, then
\[m\ddot x = -\dot U (x)\]
And by the taylor expansion this is
\[m\ddot x = -\ddot U(x_0)(x) + O(x^2)\]
For small $x$ we can drop the quadratic term so get
\[\ddot x = -\omega_0^2 x\]
where $\omega_0^2 = \frac{\ddot U(x_0)}{m} > 0$ which is positive if $x_0$ is a minimum and negative for a maximum. 

We have $x = A \cos (\omega t)$, $\dot x = -\omega_0^2 x$. If we set $\dot x = y$ then $\dot y = -\omega_0^2 x$ and $\frac{dy}{dx} = \frac{-\omega_0^2 x}{y}$. Sepparating this equation and integrating gives $\frac{1}{2}y^2 + \frac{1}{2}\omega_0^2x^2 = E$ which we can interpate as the kinetic energy plus the spring energy if $\omega_0^2 = k$ for the spring constant. 

\subsection{}
This is a second order partial differential equation in time adn thus has two constants. The solutions are
$|C|\cos(\omega_0 t + \delta$ where $\delta = arg C$ or $A \cos (\omega_0 t) + B \sin (\omega_0 t)$ or $\Re (Ce^{i\omega_0 t}$ where $C = A - iB$.




\subsection{Phase Space}
In phase space we can write $\dot x = y$ and let $(x,y)$ be coordinates in phase space. So we have $\ddot x = \dot y =  -\omega_0^2 x$

We can get an energy equation independant of time by multiplying our equation by $\dot x$ to get $\dot x \ddot x = -\omega_0^2 x \dot x$ = $\frac{1}{2}y^2 + \frac{1}{2}\omega_0^2x^2 = E$ where this constant of integration is the energy and this equation gives a family of elipses parameterized by $E$




\section{Damped simple harmonic motion}
In this case we have $\ddot x + 2\beta \dot x + \omega_0^2 x = 0$ where $2\beta >0$ is the damping coefficient. This equation has solutions of the form $x(t) = Ae^{\lambda t}$ where $\lambda = -\beta \pm \sqrt{\beta^2 - \omega_0^2}$ which gives $\lambda_1, \lambda_2$ depending on the sign

\begin{enumerate}
    \item  If $\beta > \omega_0^2$ then $\lambda_i$ is real and negative - "over damped"
    \item $\beta^2 = \omega_0^2$, $\lambda_1 = \lambda_2$ "critically damped"
    \item $\beta^2 < \omega_0^2$ so $\lambda_1 = \overline{\lambda_2}$ "under damped
\end{enumerate}


In this case to get the phase space picture for $\ddot x + 2\beta \dot x + \omega_0^2 x = 0$ we let $\dot x = y$ so $\dot y = -2\beta y - \omega_0^2x$. And then the energy is 

\subsection{Over Damped}

$x(t) = Ae^{\lambda_1 t} + Be^{\lambda_2 t}$

Inital value problem: $x(0) = x_0$, $\dot x(0) = 0$.

So $x_0 = A + B$ and $A \lambda_1 + B \lambda_2 = 0$ so $A = \frac{x_0 \lambda_2}{\lambda_2 - \lambda_1}$, $B = -\frac{x_0 \lambda_1}{\lambda_2 - \lambda_1}$ which clearly implies $\lambda_1 \neq \lambda_2$ as is true in this case.



\subsection{critically Damped}
In this case the repeated root gives $x(t) = Ae^{-\beta t}$ but we need a second independent solution to fit to the inial data. We guess that it is of the form $v(t)e^{-\beta t}$. Substituting into $\ddot x + 2\beta \dot x + \omega_0^2 x = 0$ and using the fact that $\omega_0^2 - \beta^2 = 0$ we get $v'' = 0$ or $v = A + Bt$. 
$x(t) = (A+Bt)e^{-\beta t}$


Initial value problem:

$A = x_0$, and then we differentiate to get $-\beta A + B = 0$ so $B = \beta x_0$

\subsection{Under damped}
$x(t) = e^{-\beta t}C \cos(\sqrt{\omega_0^2 - \beta^2}t + \delta)$ (shifted frequency of $\omega_0'  = \sqrt{\omega_0^2 - \beta^2}$ times a decaying exponential).


Initial value problem: $C\cos(\delta) = x_0$ and differentiating gives $-\beta C \cos (\delta) - \sqrt{\omega_0^2 - \beta^2}\sin (\delta) =0$ so $\tan(\delta) = \frac{-\beta}{\sqrt{\omega_0^2 - \beta^2}}$, $\delta >0$ is the phase lag from dissapation. So $\cos (\delta) = \frac{\sqrt{\omega_0^2 + \beta^2}}{\omega_0}$, $C = \frac{\omega}{\sqrt{\omega_0^2 - \beta^2}}$



Phase space interperatation

$\dot{\m{x \\ y}} = \m{0 & 1 \\ -\omega_0^2 & -2\beta} \m{x \\ y}$. Since $\MM$ is constant we guess the solution is $\dot{\m{x \\ y}} = \dot{\m{x_0 \\ y_0}}e^{\lambda t}$ which meands we must have $(\lambda \mathbb{I} - \mathbb{M})\m{x_0 \\ y_0} = 0$






\section{Conservation of Energy}

Work done by the force moving a particle from $\vec{r}$ to $\vec{r + dr}$ is 
\[dw = f\cdot \vec{dr} = m\vec{\dot{v}} \cdot \frac{dr}{dt}dt = m\vec{v} \cdot dv = d(\frac{1}{2}mv^2) \]
where $v^2 = \vec{v} \cdot \vec{v}$. This implies 
\[W = \int_{\vec{r_0}}^{\vec{r_1}}d(\frac{1}{2}mv^2 = T_1 - T_0 \]





We 





$\vec{r}(t) = r(t)\hat{e_r}(t)$, with $\hat e_{\theta}$ pointing in the direction of motion.
\[e_r = cos(\theta)\hat i\]




\section{Conservation of angular Momentum}
With respect to some point, $\vec{L} = \sum_{\alpha}\vec{r}_{\alpha} \times \vec{p}_{\alpha} = \sum_{\alpha}\vec{r}_{\alpha} \times  m_{\alpha}\vec{\dot{r}}_{\alpha}$ which we can write as 
\[\vec{L} = \sum_{\alpha} (\vec{R} + \vec{r'}_{\alpha}) \times m_{\alpha}(\vec{\dot{R}} + \vec{\dot{r}'}_{\alpha})\]
where $R$ is the position of the center of mass of the system. We expand and simplify, using that $\sum_{\alpha}m_{\alpha}\vec {r'}_{\alpha} = 0= \sum_{\alpha}m_{\alpha}\vec {\dot{r}'}_{\alpha}$ because this is the center of mass calculated with repect to the center of mass (since we defined $\vec{r'}$ as the vector pointing from the center of mass to a particle). Then the expression becomes
\[\vec{L} = \vec{R} \times \vec{P} + \sum_{\alpha}\vec{r'}_{\alpha} \times \vec{p'}_{\alpha}\]
which is the angular momentum of the center of mass plus the interal angular momentum. 
Taking the derivative we get
\[\vec{\dot{L}} = \sum_{\alpha}\vec{r'}_{\alpha} \times \vec{p'}_{\alpha}\]
Question: Why does the other term vanish?




\section{Forced Oscilations}

$\ddot x + 2\beta \dot x + \omega_0^2 x = f(t)$ where $f(t)$ is the driving force of the system. Suppose that $f(t) = C \cos (\omega t)$ where $\omega$, the driving frequency, is not necesseraly $\omega_0$. 

Solution: 




\section{Two Body Problems}

Consider two masses, $m_1$, $m_2$, approximated as point particles, at positions $r_1$, $r_2$ from the origin $O$, where $F_{12}$ and $F_{21}$ are the only forces and they are conservative and central.
Since the forces are conservative we can have a potential function. For gravity, this means the force is $\frac{Gm_1m_2}{|\vec{r_1} - \vec{r_2}|^2}$ and the potential is  $-\frac{Gm_1m_2}{|\vec{r_1} - \vec{r_2}|}$

We can see that for a system with a central focre that is translation invarient, $U(\vec{r_1},\vec{r_2})$ can actually be written as $U(|\vec{r_1}-\vec{r_2}|)$, so only the distance between the objects matters. We can then use $\vec{r} = \vec{r_1} - \vec{r_2}$ as the position of $1$ relative to $2$, where only the magnitude matters.


In general we want to choose the origin to be the center of mass
\[\vec{R} = \frac{r_1m_1 + r_2m_2}{m_1 + m_2}\]
The total momentum of this system is the same as if all of the mass were concentrated at the center of mass moving along with it - $\vec{P} = M\vec{\dot R}$ Since the total momentum is constant, $\vec{\dot R}$ is constnat so we can choose an inertial frame moving along with the CM, where the total momentum is $0$ in this frame.

In the case that $m_2 >>> m_1$, then the center of mass is approximately at $m_2$, which is moving very slowly, so we can think of it as roughly a single body problem.

The potential energy is in terms of $r$, and additionally, $\vec{r_1} = \vec{R} +\frac{m_2}{M}\vec{r}$ and $\vec{r_2} = \vec{R} -\frac{m_1}{M}\vec{r}$ so the total kinetic energy of the system is 
\[\frac{1}{2}(m_1\vec{\dot{r_1}}^2 + m_2\vec{\dot{r_2}}^2)\]
which expands and simplifies to 
\[\frac{1}{2}(M\vec{\dot{R}}^2 + \mu \vec{\dot{r}}^2)\]
if we let $\mu = \frac{m_1m_2}{m_1 + m_2}$ be the "reduced mass"

\subsection{Angular Momentum}

In the CM frame of reference the total angular momentum is
\[L = \vec{r_1}\times\vec{\dot{r}_1}m_1 + \vec{r_2}\times\vec{\dot{r}_2}m_2\]
In the CM frame with the center of mass at the origin (R = 0), we get $\vec{r_1} =\frac{m_2}{M}\vec{r}$ and $\vec{r_2} = -\frac{m_1}{M}\vec{r}$ and substituting in we get
\[L = \vec{r_1}\times\vec{\dot{r}_1}m_1 + \vec{r_2}\times\vec{\dot{r}_2}m_2 = \dot{\vec{r}} \times \vec{r} \mu\]
Since the angular momentum is constant, this means that $\dot{\vec{r}} $ and $ \vec{r}$ remain in the same plane so the motion reduces to two dimentions.

Question: why does this imply the same plane (could theta be compensated for by differences in the magnitudes)


\subsection{Equations of motion}

Now that we know we are working in two dimentions we write out the equations of motion. The first is from the fact that the foces are central so angual momentum is conserved giving
\[\mu r^2\dot{\theta} = l = \text{constant}\]
and the second is from the radial coordinate of $F  = ma$ (using polar coordinates) and using $\mu$ instead of the two masses, giving
\[\mu r(\dot{\theta})^2 - \frac{dU}{dr} = \mu\ddot{r}\]

\subsection{The equavalent one dimentional problem}

Since $\mu r^2\dot{\theta} = l$ we can use this to eliminate $\dot{\theta}$ to get
\[\mu\ddot{r} =- \frac{dU}{dr} + F_{cf} \]
where $F_{cf}$ is a ficticious "centrifugal" force $F_{cf} = \frac{l^2}{\mu r^3}$. 



We can express this in terms of a "centrifugal potential energy" as $f_{cf} = -\frac{d}{dr}\frac{l^2}{\mu r^3} = -\frac{dU_{cf}}{dr}$ where this potential energy is defined as $\frac{l^2}{2\mu r^2}$





\end{document}